\documentclass[a4paper, 12pt]{report}

% Packages
\usepackage[protrusion=false]{microtype}
\usepackage{setspace}

% Language Package
\usepackage[italian]{babel}
\usepackage[italian]{cleveref}
\usepackage[toc,page]{appendix}

% Environments
\newenvironment{packed_enum}{
\begin{enumerate}
        \setlength{\itemsep}{1pt}
        \setlength{\parskip}{0pt}
        \setlength{\parsep}{0pt}
}{\end{enumerate}}

\newenvironment{packed_itemize}{
\begin{itemize}
        \setlength{\itemsep}{1pt}
        \setlength{\parskip}{0pt}
        \setlength{\parsep}{0pt}
}{\end{itemize}}

% Initialization
\title{Relazione Progetto HPC 2023}
\author{Michele Montesi \\
        Matricola: 0000974934 \\
        E-Mail: michele.montesi3@studio.unibo.it}

\date{\today}

\begin{document}
\maketitle
\tableofcontents

\chapter{Introduzione}
\begin{sloppypar}
La presente ricerca presenta lo sviluppo di due versioni parallelizzate del software \texttt{sph.c}, 
implementate utilizzando la libreria \texttt{OpenMP} e la libreria \texttt{MPI} rispettivamente. 
L'obiettivo della ricerca è quello di valutare i vantaggi prodotti dall'utilizzo della programmazione multi-processore 
confrontando poi le prestazioni delle due versioni su macchine con architetture differenti.
\end{sloppypar}

\bigskip

\begin{sloppypar}
\noindent
Durante la fase di test, le due versioni sono state eseguite su due macchine:
\begin{packed_enum}
  \item Computer multipurpose: \texttt{AMD Ryzen 9 3900} a 12 core
  \item Server ISI-Raptor: \texttt{Intel(R) Xeon(R) E5-2603 v4} a 12 core
\end{packed_enum} 
L'analisi dei risultati ha evidenziato che l'architettura delle macchine ha un impatto significativo sulle 
prestazioni del software, influenzate dalle risorse hardware e dalle modalità di parallelizzazione utilizzate.
\end{sloppypar}

\chapter{Versione OpenMP}

\chapter{Versione MPI}

\chapter{Conclusioni}

\end{document}